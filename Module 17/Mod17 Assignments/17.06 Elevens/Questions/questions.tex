\documentclass[12pt]{article}
\usepackage[margin=1in]{geometry}

\author{Tushaar Kamat}
\title{Elevens Lab Questions}

\begin{document}
\maketitle

\textbf{Activity 2}
\begin{enumerate}
  \item A card holds the values of a single card using its rank, suit, and
    pointValue, while a deck holds many cards of different types that can be
    dealt individually.
  \item The deck would contain 6 cards. 
  \item 
\begin{verbatim}
String[] ranks = { "ace", "two", "three", "four", "five", "six", "seven", 
                       "eight", "nine", "ten", "jack", "queen", "king" };
String[] suits = { "hearts", "diamonds", "spades", "clubs" };
int[] pointValues = { 11, 2, 3, 4, 5, 6, 7, 8, 9, 10, 10, 10, 10 };
\end{verbatim}
  \item The order does not matter, as long as the rank array order matches with
    the pointValue array order. Since the decks are eventually going to get
    shuffled, it doesn't matter what order the cards are in to start. 
\end{enumerate}

\textbf{Activity 3}
\begin{enumerate}
\item
\begin{verbatim}
public static String flip() {
    Random rand = new Random();
    int randInt = rand.nextInt(3);
    if (randInt < 3) 
        return "heads"
    else 
        return "tails"
}
\end{verbatim}
\item
\begin{verbatim}
public static boolean arePermutations(int[] a1, int[] a2) {
    matchCounter = 0;
    for (int i = 0; i < a1.length; i++) {
        for (int j = 0; j < a2.length; j++) {
            if (ai[i] == a2[j])
                matchCounter++;
        }
    }
    if (matchCounter == a1.length && matchCounter == a2.length)
        return true;
    else 
        return false;
}
\end{verbatim}
\item 0, 2, 1
\end{enumerate}

\textbf{Activity 6}
\begin{enumerate}
\item 5 of spades with 6 of clubs, 6 of clubs with 5 of clubs. 
\item If the deck is empty, the last three cards have to be J, Q, and K since
  there are an even number of non face cards, so it would be impossible to have
  an odd number of left if they are only removed two at a time. Furthermore,
  face cards are only removed three at a time, so they can only exist in
  multiples of three.
\item There is little strategy involved, except possibly prioritizing J Q K
  combos over normal pairs since they take up more board space.
\end{enumerate}


\textbf{Activity 7}
\begin{enumerate}
\item Instance variables needed would be 9 spaces, a deck of cards, and a
  discard pile. 
\item \begin{itemize}
      \item Create and shuffle a standard deck
      \item Deal 9 cards into an array.
      \item Check for J Q K triplets and replace.
      \item Check for pairs that add up to 11 and replace.
      \item Repeat previous two steps until either no options are available, in
        which you lost, or no cards are left, int which you win.
      \end{itemize}
\item No, it just contains the code to set up and access the board.
\item \begin{enumerate}
      \item It is called in the constructor.
      \item isLegal and anotherPlayIsPossible.
      \item 0, 1, 3, 6, 7.
      \item
      \begin{verbatim}
public static printCards(ElevensBoard board) {
    List<Integer> cIndexes = board.cardIndexes();
    for (int card : cIndexes)
        System.out.println(card);
}
      \end{verbatim}
      \item anotherPlayIsPossible, since it needs to look at all of the cards on
        the board, while isLegal only looks at the selected cards given as the
        parameter. 
      \end{enumerate}
\end{enumerate}


\textbf{Activity 8}
\begin{enumerate}
\item Both \textit{Elevens}, \textit{Thirteens}, and \textit{Tens} all share the
  same basic game structure of choosing pairs of cards that add up to a certain
  number, and replacing them with new cards from the deck. They also share the
  same winning and losing conditions, with victory being defined as having no
  cards left in play and loss being defined as having no available moves. The
  games differ on their methods of getting rid of high-value/face-cards, and
  their win rates.
\item The Board instance variables are initialized by the constructor in the
  ElevensBoard class, and are then passed on to the instance variables
  initialized by the superclass. 
\item The abstract methods do not cover all of the differences between
  \textit{Elevens}, \textit{Thirteens}, and \textit{Tens}, since each game also
  needs specific methods to check for pairs and to check for combinations of
  face-cards/high-value cards.
\end{enumerate}

\textbf{Activity 9}
\begin{enumerate}
\item Size is calculated the same way no matter the actual size of the board;
  even if the outputs differ, the actual mechanism is the same. 
\item The selection and replacing of the cards is the same between games, so
  there is no need for an abstract method. 
\item Since isLegal and anotherPlayIsPossible are abstract methods, they could
  still be called polymorphically. The main difference and disadvantage of using
  an interface is the fact that interfaces cannot use instance variables,
  constructors, or non-abstract methods.
\end{enumerate}

\end{document}
